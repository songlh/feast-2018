\section{Introduction} 
\label{sec:intro}

% Background

To facilitate communication with external entities and coordination with
internal operations, private networks (\eg, data centers and industrial control
networks) run a wide range of conventional protocols (\eg, Secure Socket Layer
\texttt{SSL}~\citep{tls-wiki} and Lightweight Directory Access Protocol 
\texttt{LDAP}~\citep{LDAP}). 
As a result, they are typically vulnerable to not only the security loopholes
residing in the protocol but also the DDoS attackes resulting from protocol
abuses~\citep{LDAP-abuse1,LDAP-abuse2,NTP-abuse}.

To safeguard such private networks, network administrators typically adopt
firewalls~\citep{firewall} to detect intruders. 
However, they provide only limited security
guarantee to the daily operations in private networks~\citep{firewall-limitation}. 
On the one hand, this is
because private networks use public protocols both internally and externally,
and it is sometimes difficult for firewalls to characterize the abnormal
behaviors of public protocols. On the other hand, this is due to the fact that
firewalls are typically ineffective when the attack is launched internally.

From the perspective of network defenses, we believe the best practice for
addesssing this issue is to customize protocols so that they can be used for
internal and external communications differently. By customizing the dialog of
\texttt{LDAP} protocol and using it only for internal communications or
operations, for example, we could augment a private network with the ability to
drop off the internal packets different from the customization, making the
network naturally resistant to the DDoS reflection attack resulting from the
abuse of conventional \texttt{LDAP}. However, protocol customization could be
extrememely difficult and time-consuming. Given a complicated communication
protocol, it generally requires a software developer putting a large amount of
manual efforts to ensure his customization does not introduce logic
incorrectness nor runtime failures.

% Our research

Inspired by this, we propose a protocol customization techniques to diversify
protocols commonly adopted in private networks. More specifically, we plan to
design and develop a full stack of technical approaches that could automate
protocol modification and thus generate various dialogs for a target protocol.
Since a network protocol typically can be modeled as a set of finite state
machines involved in communication, technically speaking, we will first develop
an effective mechanism to identify the implemenation pertaining to the state
machines responsible for message exchanging. Then, we will utilize and design
static program analysis techniques to analyze protocol implementations,  extract
state machines and fully restore the communication dialogs they involved.

Since the ultimate goal of this research project is to generate various protocol
dialogs, we will further explore various approaches to mutate a communication
dialog and identify those mutation strategies that do not jeoparidize the
validity of a target protocol. We intend to implement the strategies  identified
through an automated tool. Thus, we will also design and develop various program
synthesis techniques. To be more specific, we will research how to utilize
various program analysis techniques to perform code removal as well as code
generation. In the rest of the proposal, we will mainly descirbe how we plan to
design and develop these technical mechanisms, and discuss the technical
challenges we have to address. In addition, we introduce the research works that
we have already conducted.






% Private network always becomes the target of an attack, which extrernal
% intruders probe the networks and attempts to exploit or conduct Dos. To
% address this problem, traditional solution rely upon firewall to detect
% intruder. While it is effective, it does not guarantte to block external
% intrusion because the limitation.  Network intrusion largely rely upon the
% signatures. But some attack does not have signatures. Therefore, it sometimes
% fails to flag a intrusion or mistakenly tag a legit connect as malicious.

% To address this issue, we believe one solution to safegaurd private network is
% to use private protocol because (1) identifying intrusion detection in a
% private network without firewall (2) could potentially reduce the code bases
% -- tightening constraint -- making software resistant to known vulnerabilities
% (3) could reduce the network overhead -- merging network traffic. However,
% customizing a protocol to a private dialog needs a lot of manual efforts and a
% large amount of expertises. To make it worse, it is difficult to ensure the
% customized protocol is still valid.





