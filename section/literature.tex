\section{Related Work}
\label{sec:literature}

As we mentioned above, the proposed research includes the extraction of state
machines as well as protocol dialect generation. To the best of our knowledge,
there is no previous research that focuses on protocol dialect generation. As a
result, the research most related to our own is protocol reverse engineering, in
which one line of work lies in recovering state machines from target protocols.
In this section, we briefly summarize these previous research works and discuss
the their limitations. It should be noticed that we do not include those
protocol reverse engineering works that focus only on packet format recovery
(\eg,~\citep{AutoFormat, Polyglot, Dispatcher, ucsb-packet, wangzhi, Junghee,
Junghee2,Junghee3,Discoverer,Tupni}).

Cho et al. proposed two technical approaches to infer state machines from a
target protocol. In~\citep{botnet-inference}, they utilized parallelization  and
caching to improve \texttt{L*} algorithm~\citep{L1, L2} and then used it to
perform state machine  extraction. In~\citep{MACE}, they inferred finite state
machines using concolic execution~\citep{dart,cute}  along with \texttt{L*}
algorithm. In~\citep{Prospex}, Comparetti et al introduced \texttt{Prospex} to
extract state machine. Technically, they are similar to the works proposed
in~\citep{botnet-inference, MACE}, which restores message format using machine
learning and then infer state machines using \texttt{L*}. While the techniques
above all demonstrated the capability of extracting state machines from a
protocol, they cannot be adopted in our research work. In the context of dialect
generation, we need to associate the state machine with the code fragments
indicating their implementation. However, the previous works all treat a target
protocol as a blackbox. Thus, they all lack the ability to provide guidance for
pinpointing the state machine implementation.