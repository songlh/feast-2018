\section{Related Works}
\label{sec:related}

There are two goals to reverse engineer a protocol, 
one is to identify formats of packets exchanged through the protocol, 
and the other is to extract state machines which maintain valid sessions of the protocol. 
Existing techniques can be categorized, based on their different goals. 



\subsection{Packet Format Extraction}

To extract packet formats, 
techniques can be built based on dynamic taint analysis~\citep{AutoFormat,Polyglot,Dispatcher,ucsb-packet,wangzhi}, 
static analysis~\citep{Junghee} or analyzing network traces~\citep{Discoverer}. 

%{\textbf{\underline{Dynamic Analysis}}
{\underline{\textit{Dynamic techniques}}}
first generate execution traces using taint analysis~\citep{dawn-taint,taint1,taint-sosp,taint-micro}, 
which will mark data received from network with taint information 
and propagate this information to any instruction that operates on tainted data.
Dynamic techniques then analyze the execution traces to figure out packet formats. 
The intuition behind these techniques is that how protocol implementations 
process received packets indicates packet formats. 

Polyglot~\citep{Polyglot} models received packets in a flat structure, 
and identifies length fields, pointer fields, field separators, 
and protocol keywords through heuristics. 
~\citet{ucsb-packet} consider a untainted value as a potential delimiter, 
when the untainted value is compared with a tainted input byte, 
and label the input byte with the delimiter.
After that, a received packet will be splitted into nested intervals by delimiters, 
which represents nested fields. 
AutoFormat~\citep{AutoFormat} hierarchically identify packet fields based 
on their access contexts, such as call stacks or the addresses of instructions. 
There are many library calls and instructions whose arguments or outputs contain network semantic information, 
such as IP addresses, port numbers and timestamp.
Dispatcher~\citep{Dispatcher} leverages these system calls and 
instructions to attribute semantic information to each identified packet field. 
Dispatcher also tries to figure out fields of output packets. 
Dispatcher models an output packet as a buffer, 
which is recursively constructed from other small buffers. 
Each small buffer represents a packet field. 

{\underline{\textit{Static techniques}}} 
apply control flow and data flow analysis to infer output file or packet formats. 
FFS/x86 (File-Format Extractor for x86) uses hierarchical finite state machine (HFSM)~\citep{HFSM1, HFSM2} 
to represent output data format~\citep{Junghee,Junghee2}.
One node in HFSM represents an output library call or a callsite to a sub-FSM. 
After generating initial HFSM, 
FFS/x86 annotates each node with possible values to be written out 
and the size of the data the node represents. 
In the end, FFS/x86 simplifies HFSM by merging nodes annotated with uncertain values or sizes.
Although the idea of FFS/x86 is general, the implementation is tied to x86 instruction set. 
In their following works, TSL system~\citep{Junghee3} 
is designed to generate analyzers targeting different instruction sets. 


{\underline{\textit{Network trace analysis techniques}}}  
infer formats based on multiple collected packets. 
Discoverer~\citep{Discoverer} first breaks each packet into text tokens and binary tokens, 
and conduct initial clustering for all packets.
Tokens represent potential packet fields.
Then, Discoverer recursively splits each cluster based on 
identified format distinguisher (FD) tokens.  
FD token is a type of fields, 
which can be used to differentiate formats of subsequent parts. 
Finally, Discoverer will merge similar clusters to eliminate cases, 
where packets of the same format are classified into different clusters. 
Tupni~\citep{Tupni} combines dynamic taint analysis and trace analysis. 
After generating packet format for every packet, 
Tupni will merge all packet formats to drive a more complete specification. 

Although techniques to extract packet formats are useful in guiding black-box fuzzing, 
deep packet inspection and packet rewriting, 
they do not identify protocols’ state machines and cannot provide any guidance 
in protocol subsetting and dialect generation. 


\subsection{Protocol State Machine Extraction}

Existing techniques~\citep{botnet-inference,MACE,Prospex} extract state machines through analyzing network traffics in two steps.
First, these techniques apply an abstract function to change concrete packets into a set of abstract messages.
Second, L* algorithm~\citep{L1,L2}  is applied on network traffics represented by abstract messages to generate state machines.   

L* is an online algorithm~\citep{L1,L2} to infer a finite state machine 
based on queries to the finite state machine and its responses. 
L* assumes the finite state machine to be inferred will answer queries in a deterministic way. 
It also assumes that there is an oracle which will tell whether the current learned state machine is equivalent to the real one.
If not, the oracle will generate a counterexample, which L* can use to continue the inference process. 
The complexity of L* is polynomial in the number of states and the size of input alphabet. 


~\citet{botnet-inference} design several optimizations when applying L* algorithm, 
such as parallelization and caching, and also demonstrate that 
inferred state machines can be used as a formal basis to study and defeat botnet
such as identifying the weakest link in a botnet protocol and proving the existence 
of communication channels between botnet masters.  
MACE~\citep{MACE} incrementally infers state machines by combining 
L* algorithm and concolic exploration~\citep{dart,cute}. 
In each iteration, MACE use a sequence of input packets to make the monitored protocol 
implementation get to a known state, and then use concolic execution to explore new states.  
Compared with concolic execution, MACE is less likely to be stuck in local state sub-spaces. 
Prospex~\citep{Prospex} starts state machine inference from clustering packets, 
based on packet field similarity, execution similarity and output similarity. 
Then, Prospex leverage an abstract function to assign a type to each cluster. 
A session can be represented as a sequence of packet types. 
L* algorithm is applied to infer a state machine which can take the sequence as input. 

Although these techniques are useful, 
they largely treat analyzed protocol implementation as a black box, 
and their identified protocol states and message types 
do not correspond to specific codes in the protocol implementation.
Therefore, extracted state machines cannot provide any guidance for protocol subsetting and dialect generation. 
Since all these techniques are based on collected network traces, 
which may not cover all protocol states, 
they cannot guarantee inferred state machines are complete. 
